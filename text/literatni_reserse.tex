\chapter{Literární rešerše}

\section{Principy pořizování rentgenových snímků}
\label{sec:principy}
Pořizování rentgenových snímků je prováděno pomocí zdroje záření, rentgenovaného objektu a detektoru rentgenového záření. Rentgenové záření jsou elektromagnetické vlny s~vlnovou délkou od \SI{10}{\nano\meter} do \SI{1}{\pico\meter} jejichž enerigie se nejčastěji pohybuje od \SI{1}{\kilo\eV} do \SI{200}{\kilo\eV}. \cite{AstroNuklFyzika-JadRadFyzika}

Jak již bylo zmíněno výše, scéna pro pořizování rentgenových snímků (\cref{fig:xray-scene}) se skládá ze zdroje rentgenového záření, rentgenovaného objektu a detektoru rentgenového záření. Zdroj rentgenového záření ozařuje elektromagnetickým vlněním o~vlnové délce \SI{5}{\pico\meter} až \SI{50}{\pico\meter} rentgenovaný objekt. V~závislosti na tloušťce a absorpčních vlastnostech objektu se část záření absorbuje a zbylá část záření dopadá na detektor rentgenového záření. Výstupem detektoru je poté obraz ve stupních šedi. \cite[kap.~3.2]{AstroNuklFyzika-JadRadMetody}

\begin{figure}[bh]
\includegraphics[width=\textwidth]{xray-scene}
\caption{Scéna pro pořizování rentgenových snímků.}
\label{fig:xray-scene}
\centering
\end{figure}

\subsection{Vznik rentgenového záření}
\label{sec:vznik-rentgenoveho-zareni}
Elektromagnetické záření, kterému se říká rentgenové záření vzniká buď při přechodu elektronů mezi vnitřními vrstvami těžších atomů -- charakteristické X-záření, nebo při dopadu a prudkém zabrzdění elektronů -- brzdné záření. \cite{AstroNuklFyzika-JadRadFyzika}

\subsubsection{Brzdné záření}
V~případě, že se akcelerovaný elektron přiblíží k~jádru atomu silné Coulombovy síly mezi jádrem atomu a letícím elektronem způsobí silné zbrzdění elektronu a změnu jeho trajektorie. Během zbrzdění rychle letícího elektronu  je jeho kinetická energie přeměňována na brzdné záření. \cite[str.~89]{Diagnostic-Radiology-Physics} Tento proces popisuje \cref{fig:bremsstrahlung-xray}.

\begin{figure}[bh]
\centering
\includegraphics[width=0.75\textwidth]{bremsstrahlung-xray}
\caption{Vznik brzdného záření při interakci rychle se pohybujícího elektronu s~atomem wolframu. \cite{the-xray-beam}}
\label{fig:bremsstrahlung-xray}
\end{figure}

Ideální spektrum brzdného záření lze popsat pomocí zjednodušeného modelu, který neuvažuje kvantovou mechaniku. Zjednodušený model uvažuje proud elektronů přibližující se k~atomu. Uvážíme-li pole v~okolí jádra atomu rozdělené na několik kruhových vrstev dle působících brzdných sil, generované brzdné záření brzděním proudu elektronů má spektrum, které odpovídá plochám těchto vrstev. Čím blíže je vrstva pole k~jádru atomu, tím větší brzdnou sílou působí na letící elektron a tím větší je energie vygenerovaného fotonu. Zároveň čím je vrstva pole vzdálenější od jádra atomu, tím větší má plochu a tím více fotonů je schopna vygenerovat. Fotony generované ve vzdálenějších vrstvách mají nižší energii vzhledem k~nižším brzdným silám působících na přibližující se elektrony. \cite[kap.~THE~X-RAY TUBE]{The-Physical-Principles-of-Medical-Imaging}

Ideální spektrum brzdného záření zjednodušeného modelu ukazuje \cref{fig:bremsstrahlung-xray-char}. Ze spektra je zřetelné, že největší energii, která se blíží kinetické energii elektronů má jen zlomek z~celkového počtu generovaných fotonů (fotony generované elektrony, které byly zabrzděny ve vrstvě nejblíže jádru).

\begin{figure}[bh]
\centering
\includegraphics[width=0.75\textwidth]{bremsstrahlung-xray-char}
\caption{Ideální spektrum brzdného záření. \cite[str.~90]{Diagnostic-Radiology-Physics}}
\label{fig:bremsstrahlung-xray-char}
\end{figure}

\subsubsection{Charakteristické záření}
Charakteristické záření vzniká při přechodu atomového elektronu z~vyšší vrstvy do nižší. Při přechodu elektron ztrácí energii, která je emitovaná jako foton charakteristického záření. Energie emitovaného fotonu odpovídá rozdílu energií vrstev mezi kterými elektron přechází. Spektrum charakteristického záření je monochromatické a odvíjí se od druhu atomu. Proces vzniku charakteristického záření popisuje \cref{fig:characteristic-xray}. \cite[str.~91]{Diagnostic-Radiology-Physics}

\begin{figure}[h]
\centering
\includegraphics[width=0.75\textwidth]{characteristic-xray}
\caption{Vznik charakteristického záření v~atomu wolframu při vyražení elektronu z~K-Vrstvy elektronem s~kinetickou energií vyšší, než vazební energie vyraženého elektronu. \cite{the-xray-beam}}
\label{fig:characteristic-xray}
\end{figure}

\subsection{Rentgenka}
Zdrojem rentgenového záření při pořizování rentgenových snímků je nejčastěji speciální vakuová elektronka (\cref{fig:xray-tube}), které je často nazývána jako rentgenka, rentgenová lampa či rentgenová trubice. \cite{AstroNuklFyzika-JadRadMetody} Rentgenku si lze představit, jako zařízení, které převádí energii elektronů na elektromagnetické záření s~odpovídající energií. Expozice a spektrum záření může být řízena nastavením parametrů rentgenky jako jsou napětí (\SI{}{\kV}), proud (\SI{}{\mA}) a doba expozice (\SI{}{\s}). \cite[str.~93]{Diagnostic-Radiology-Physics}

\subsubsection{Principy fungování rentgenky}
Energie, která je přeměňována v~rentgence na rentgenové záření a teplo je do rentgenky přiváděna proudem elektronů s~potenciální energií odpovídající napětí na vysokonapěťovém zdroji (\SI{1}{\kV} odpovídá \SI{1}{\k\eV}). 
Během průchodu elektronu rentgenkou dochází k~přeměně jeho potenciální energie na energii kinetickou, která je následně přeměněna na elektromagnetické záření a teplo. 

Kinetická energie elektronu při dopadu na anodu odpovídá napětí na vysokonapěťovám zdroji, tedy původní potenciální energii. Kinetická energie je přeměňována zbrzdění atomů při dopadu na anodu  při interakci s~atomy materiálu na brzdné a charakteristické rentgenové záření. \cite[kap.~ELECTRON ENERGY]{The-Physical-Principles-of-Medical-Imaging}
Proud elektronu je emitován při žhavení katody se záporním napětí. Množství emitovaných elektronů může být řízeno změnou proudu žhavení, tedy změnou teploty žhaveného vlákna. \cite[str.~93]{Diagnostic-Radiology-Physics} 

\paragraph{Spektrum rentgenového záření rentgenky}
popisuje \cref{fig:xray-spectre}. Při dopadu elektronů na anodu vznikají dva typy rentgenového záření--brzdné a charakteristické, jejichž obecné principy byly popsány v~kapitole \ref{sec:vznik-rentgenoveho-zareni}. Obrázek ukazuje celkem tři charakteristiky záření při napětí na elektrodách \SI{90}{\kV}:
\begin{enumerate}[label=(\alph*)]
\item Ideální spektrum brzdného brzdného záření--spektrum, které již bylo popsáno v~obrázku \ref{fig:bremsstrahlung-xray-char}.
\item Generované spektrum--reálné spektrum skládající se z~brzdného a charakteristického záření.
\item Filtrované spektrum--spektrum s~útlumem dopovídajícím \SI{2.5}{\mm}Al.
\end{enumerate}

\begin{figure}[hb]
\centering
\includegraphics[width=\textwidth]{xray-spectre}
\caption{Rentgenka se zdroji proudu a napětí. \cite[str. 93]{Diagnostic-Radiology-Physics}}
\label{fig:xray-spectre}
\end{figure}

\subsubsection{Konstrukce rentgenky}
Z~důvodu tepelného ohřevu anody po dopadu zrychlených elektronů a vysokého napětí na elektrodách musí být rentgenky oproti běžným elektronkám robustní konstrukce. Chlazení samotné anody je zajištěno její velikostí a také rotací  nebo aktivním chlazením. Rentgenky lze rozdělit kategorií podle způsobu využití a konstrukce \cite[kap. 3.2]{AstroNuklFyzika-JadRadMetody}:
\begin{itemize}
\item Rentgenky pro průmyslové ozařování a radioterapeutické použití - rentgenky s~pevnou anodou, kde je chlazení zajištěno průtokem chladícího média. U~toho typu rentgenek je častým požadavkem vysoká energie a intenzita záření. Naopak zde není potřebné zaměřování elektronů do téměř bodového ohniska. 
\item Rentgenky pro rentgenovou diagnostiku - rentgenky se soustředěním elektronů do ohniska. U~tohoto typu rentgenek se využívá rotující anody proti nadměrnému přehřívání anody v~místě ohniska.
\item Speciální rentgenky - rentgenky rozšířené o~třetí elektrodu (drátěnou mřížku umístěnou mezi katodou a anodou v~těsné blízkosti katody) sloužící k~řízení proudu protékajícího anodou. Proud je řízen napětím, které je přivedeno na drátěnou mřížku.
\end{itemize}

Generované záření ještě před tím, než opustí rentgenku musí projít skrz různé materiály, které záření filtrují. Těmito materiály mohou být například samotná anoda, materiál trubice rentgenky, chladící medium apod. Úbytek záření při průchodu těmito materiály se uvádí jako útlum ekvivalentní k~\SI{1}{\mm}Al--jednomu milimetru hliníku. Typická hodnota útlumu u~běžných rentgenek bývá od \SI{0.5}{\mm}Al do \SI{1}{\mm}Al. 

\begin{figure}[hb]
\includegraphics[width=\textwidth]{xray-tube}
\caption{Rentgenka se zdroji proudu a napětí. \cite[str. 93]{Diagnostic-Radiology-Physics}}
\label{fig:xray-tube}
\centering
\end{figure}

\paragraph{Anoda}
je součást rentgenky, kde je generováno rentgenové záření. Anoda bývá tvořena relativně velikým kusem železného materiálu, na který je přivedeno kladné napětí vysokonapěťového zdroje. Anoda jako taková plní v~rentgence dvě funkce:
\begin{itemize}
\item Převod elektrické energie na rentgenové záření.
\item Odvod tepla, které vzniká během procesu generování rentgenového záření.
\end{itemize}
Jako vhodný materiál anody lze považovat materiál, který dokáže co největší podíl elektrické energie převést na záření, čili materiál, který má vysokou efektivitu převodu (jak již bylo zmíněno přebytečná energie je přeměňována na teplo). Efektivita převodu elektrické energie na záření závisí na atomovému číslu materiálu anody (Z) a kinetické energii dopadajícího elektronu na katodu.\cite{The-Physical-Principles-of-Medical-Imaging}

Nejvíce rentgenek využívá jako materiál anody wolfram s~atomovým číslem 74. Wolfram je vhodný díky vysokému atomovému číslu a vysokému bodu tání. V~některých případech je využíváno slitiny wolframu a rhenia, která je však využívána pouze jako povrchový materiál. Zbylá část anody anody poté bývá vyrobena z~relativně lehkého materiálu, který má dobré tepelné vlastnosti. Těmito materiály mohou být například molybden nebo grafit. Výjimkou jsou anody rentgenek pro mamografii, kdy je využíváno molybdenu jako materiálu pro povrch anody.\cite{The-Physical-Principles-of-Medical-Imaging}

Anody lze dělit v~závislosti na výkonu rentgenky (\cref{fig:anodes}). Pro aplikace, kdy není potřebná vysoká energie rentgenového záření je využíváno statických anod. Tato anoda se skládá z~wolframu, který je usazen v~měděném bloku, který slouží k~odvádění tepla. Tento typ anod se využívá například dentistických nebo přenosných rentgenkách. Druhým typem anod jsou rotační anody. Anoda je připojená k~rotoru asynchronního motoru, který je umístěn přímo ve vakuové trubici rentgenky. Vinutí statoru je naopak umístěno vně rentgenkové trubice. Samotná anoda má kruhový tvar v~podobě terče se skosenými hranami na okrajích. Paprsek elektronů poté dopadá na zkosenou hranu, která je otáčena rotorem, tudíž je anoda tepelně namáhána rovnoměrně podél celého obvodu, což umožňuje generovat záření s~vyšší energií. \cite[str.~98]{Diagnostic-Radiology-Physics}

\begin{figure}[hb]
\centering
\includegraphics{anodes}
\caption{Rotační a stacionární anoda. \cite{the-xray-beam}}
\label{fig:anodes}
\end{figure}


\paragraph{Katoda}
je druhou elektrodou rentgenky, která má za úkol generování paprsku elektronů pomocí žhavícího vlákna (\cref{fig:cathode}). Katoda je připojena a záporné napětí vysokonapěťového zdroje a zároveň k~střídavému zdroji proudu, který slouží k~žhavení vlákna katody a tím i emitování elektronů.\cite[str.~93]{Diagnostic-Radiology-Physics} Velikost žhavícího vlákna ovlivňuje velikost ohniska paprsku elektronů na anodě. Čím větší je žhavící vlákno tím větší ohnisko je.

\begin{figure}[hb]
\centering
\includegraphics{cathode}
\caption{Katoda s~dvěmi žhavícími vlákny. \cite{the-xray-beam}}
\label{fig:cathode}
\end{figure}

\subsection{Detekce rentgenového záření}
Rentgenové záření může být detekováno pomocí tzv. detektorů rentgenového záření. Tyto detektory lze dělit podle způsobu záznamu rentgenového záření na analogové a digitální. Analogové detektory využívají principu záznamu rentgenového záření v~podobě snímku na film, kdežto digitální zaznamenávají snímky jako digitální snímek. Digitální detektory lze dělit dle toho, zda je snímek digitalizován přímo během procesu ozařování (\zkratka{zk:DR}) nebo až po procesu ozařování (\zkratka{zk:CR}). Systémy patřící do \zkratka{zk:DR} lze rozdělit dle toho, zda snímají přímo rentgenové záření nebo rentgenové záření převedené na světelné záření na \zkratka{zk:DR} s~přímým a nepřímým převodem rentgenového záření. V~případě digitálních detektorů je často zmiňována velikost pixelů. Touto velikostí je myšlena plocha snímače rentgenové nebo světelného záření detektoru, který dopovídá jednomu pixelu.

\subsubsection{Parametry detektorů}
\paragraph{Velikost pixelu, velikost detektoru a rozlišení}
jsou jedněmi ze základních parametrů digitálních detektorů. Velikost pixelů, respektive velikost snímací plochy snímače odpovídající pixelu a vzdálenost mezi nimi ovlivňují především prostorové rozlišení detektoru. \cite[str.~682]{Advances-in-Digital-Radiography} Celková velikost detektoru poté ovlivňuje jak velkou scénu lze snímat (flat-panel detektory), případně jaký způsobem je scéna snímána (snímání pohybujícího se rentgenovaného předmětu nebo scény pomocí řádkového CCD detektoru např. na výrobní lince). Rozlišení digitálních detektorů odpovídá počtu pixelů/snímačů a je udáváno v~počtech pixelů na šířku a výšku.

\paragraph{Prostorové rozlišení}
představuje minimální vzdálenost mezi dvěma kontrastními objekty, při které lze tyto dva objekty pomocí detektoru od sebe rozlišit. \cite[str.~682]{Advances-in-Digital-Radiography} Jinými slovy je to vzdálenost mezi dvěma objekty na rentgenovém snímku, při kterém nám tyto dva objekty nesplývají v~jeden. Jak již bylo zmíněno, maximální prostorové rozlišení digitálních detektorů vychází z~minimální velikosti pixelů těchto detektorů. Vezmeme-li v~potaz Nyquistův teorém o~vzorkovací frekvenci, maximální dosažitelné rozlišení v~případě, že má pixel velikost $a$ je nižší než $a/2$. 

Mimo minimální velikosti pixelů prostorové rozlišení ovlivňuje také rozptyl uvnitř detektorů. Například selenové polovodičové detektory s~přímým převodem mají lepší prostorové rozlišení než detektory s~nepřímým převodem s~nestrukturovanou scintilační vrstvou.

\paragraph{Modulační přenosová funkce -- Modulation Transfer Function (MTF)}
úzce souvisí s~prostorovým rozlišením detektoru. \zk{zk:MTF} jako jednotka je vhodná pro určení pravého nebo efektivního rozlišení detektoru, protože zahrnuje míru rozostření a změny kontrastu v~v~určitém rozsahu prostorových frekvencí.\zk{zk:MTF} lze definovat jako podíl kontrastu výstupního signálu detektoru ku kontrastu vstupního signálu detektoru. \cite{Modulacni-Prenosova-Funkce} Jinými slovy \zk{zk:MTF} říká, jak se  přemění kontrast vstupního signálu detektoru pro určitou prostorovou frekvenci vstupního signálu na kontrast výstupního signálu. \cite[str.~682]{Advances-in-Digital-Radiography} Prostorová frekvence vstupního signálu je uváděna v~cyklech na \SI{}{\mm}.

\paragraph{Dynamický rozsah}
detektoru vyjadřuje odezvu výstupu detektoru na jeho vstup. Dynamický rozsah je popsán funkcí dávky rentgenového záření jejíž výstupem jsou hodnoty v~rozsahu od minimální do maximální možné hodnoty výstupu detektoru. Například v~případě filmových systémů je tento rozsah poměrně malý, tudíž se může stát, že nesprávně zvolenou expoziční dobou dochází k~přeexponování snímku. Naopak u~digitálních detektorů je dynamický rozsah detektoru poměrně vysoký, čímž riziko přeexponování snímku klesá, ale naopak například v~lékařské praxi může docházet k~vystavování pacienta zbytečně velké dávce rentgenového záření. Závislosti výstupů filmových a obecných digitálních detektorů na dávce rentgenového záření popisuje \cref{fig:dynamic-range}. \cite[str.~682]{Advances-in-Digital-Radiography}

\begin{figure}[ht]
\centering
\includegraphics{dynamic-range}
\caption{Závislost výstupu detektorů na dávce rentgenového záření. \cite[str.~682]{Advances-in-Digital-Radiography}}
\label{fig:dynamic-range}
\end{figure}

\paragraph{Detekční kvantová účinnost -- Detective Quantim Efficiency (DQE)}
určuje účinnost detektoru při převodu energie fotonu rentgenového záření na výstupní signál. \zk{zk:DQE} lze popsat jako poměr poměru signál-šum -- signal-to-noise ratio (SNR) výstupního signálu ku SNR vstupního signálu. SNR výstupního a vstupního signálu je funkce závislá na prostorové frekvenci. \zk{zk:SNR} a tím i \zk{zk:DQE} jsou závislé na frekvenci, \zk{zk:MTF} a materiálu detektoru. 

Čím vyšší hodnota \zk{zk:DQE} je, tím menší dávka rentgenového záření je potřeba k~získání snímku o~stejné kvalitě oproti detektoru s~nižší hodnotou \zk{zk:DQE}. Tato závislost lze také vyjádřit tak, že při stálé době expozice detektor s~vyšší \zk{zk:DQE} detekuje kvalitnější snímek oproti detektoru s~nižší hodnotou \zk{zk:DQE}. Hodnota \zk{zk:DQE} ideálního detektoru je hodnota 1. V~tomto případě by byl detektor schopen přeměnit veškerou energii rentgenového záření na obrazovou informaci.

\subsubsection{Filmové systémy}
Pro získávání snímků pomocí filmu je využíváno filmových systému. Tento systém využívá filmu, který je citlivý na světelné záření. Pro získání rentgenového snímku je třeba vložit film do rentgenového zařízení a poté film vyvolat. Film bývá vložen do plastové kazety, uvnitř kterých je fosforová vrstva, která přeměňuje rentgenové záření na světelné záření, které lze detekovat filmem vloženým do kazety. Výhodou filmových systémů je, že film jako takový má vysokou citlivost a zároveň se při detekci na rozdíl od digitálních systémů neztrácí žádná viditelná informace. \cite[str.~155]{Diagnostic-Radiology-Physics} Filmové systémy jsou využívány jak ve zdravotnictví, tak v~průmyslu, nicméně jejich využití je spíše na ústupu. Vzhledem k~povaze této práce se jimi práce nebude více zabývat.

\subsubsection{Digitální systémy}
Čím dál častěji filmové systémy nahrazují systémy digitální. tyto systémy lze rozdělit do dvou kategorií na systémy pro nepřímou radiografii -- Computed Radiography (\zk{zk:CR}) a přímou radiografii -- Direct Radiography  (\zk{zk:DR}). \zk{zk:CR} vychází z~klasických filmových systému u~kterých je místo filmu využíváno speciálních fosforových fólií. Proces digitalizace je pak prováděn ve speciálním zařízení.
\zk{zk:DR} na rozdíl od \zk{zk:CR} převádí záření na digitální snímek přímo, nicméně samotný proces převodu může být přímý (záření je převáděno snímači na elektrickou veličinu) nebo nepřímý (záření je nejprve převedeno na světelné záření, které je poté převáděno na elektrickou veličinu). \cite[str.~676]{Advances-in-Digital-Radiography}

\paragraph{Nepřímá radiografie (\zk{zk:CR})}
využívá technologie, která je velice podobná filmovým systémům. Paměťové fólie jsou tvořeny citlivou vrstvou tvořenou fosforovými krystaly. \cite[str.~676]{Advances-in-Digital-Radiography}. 

Proces ukládání informace a její digitalizaci popisuje \cref{fig:cr}. Energie dopadajícího záření je absorbována a dočasně (řádově několik hodin) uložena v~paměťové fólii přechodem elektronů atomů fosforových krystalů do vyšších energických hladin (metastabilních). Digitalizace uložené energie ve fólii poté probíhá v~zařízení, které dokáže pomocí laserového paprsku uvolňovat elektrony z~metastabilních hladin do vodivostního pásu. Při uvolňování elektronů dochází k~emitování světelného záření, které je poté možné měřit fotonásobičem připojeným na AD převodník. \cite[str.~677]{Advances-in-Digital-Radiography}. Výhodou tohoto systému je snadná integrace do stávajících filmových systémů. Nevýhodou může být vysoká časová náročnost vyvolávání snímků, nízké prostorové rozlišení. \cite[str~209]{Diagnostic-Radiology}

\begin{figure}[ht]
\centering
\includegraphics{cr}
\caption{Proces získání digitálního snímku při nepřímé radiografii. \cite[str.~677]{Advances-in-Digital-Radiography}}
\label{fig:cr}
\end{figure}

\paragraph{Přímá radiografie (\zk{zk:DR})}
\zk{zk:DR} je možné dále dělit podle způsobu převodu rentgenového řazení na \zk{zk:DR} s~přímým  převodem a \zk{zk:DR} s~nepřímým převodem. Při přímém převodu je rentgenové záření snímáno přímo, naopak při nepřímém převodu je záření převáděno zpravidla pomocí scintilační vrstvy na záření světelné, které je poté snímáno fotocitlivými snímači. Rozdíl mezi přímým a nepřímým převodem rentgenového záření popisuje obrázek \cref{fig:direct-indirect-tft}.

\begin{figure}[ht]
\centering
\includegraphics{direct-indirect-tft}
\caption{Polovodičový TFT detektor s~nepřímým (A) a přímým (B) převodem rentgenového záření \cite[str.~511]{Radiation-Detection-and-Measurement}}
\label{fig:direct-indirect-tft}
\end{figure}

\paragraph{DR s~přímým převodem}
využívají polovodičové vrstvy ze selenu (Se), který dokáže převádět fotony rentgneového záření na elektrickou energii. \zk{zk:DR} systémy s~přímým převodem mohou využívat buď selenové foto-válce s~AD převodníkem nebo selenové polovodičové vrstvy spojené s~vrstvou tenkovrstvých tranzistorů -- Thin Film Tranzistors (\zk{zk:TFT}). \cite[str.~678]{Advances-in-Digital-Radiography} Výhodou těchto systémů je především vysoké prostorové rozlišení. Díky nízkému atomovéu číslu selenu je u~těchto detektorů poměrně nízká absorpce rentgenového záření. Díky tomuto faktu je systémů s~přímým převode využíváno spíše v~mamografii. \cite[str~210]{Diagnostic-Radiology}

Systémy, které využívají foto-válců otáčí samotným válcem, na který dopadá rentgenové záření, které se přeměňuje na náboj na povrchu válce. Náboj na povrchu otáčejícího se válce je poté vyčítán pomocí AD převodníku. \cite[str.~677]{Advances-in-Digital-Radiography}. Tento princip je založen na stejném principu jako foto válec laserových tiskáren.
 
Na podobné principu pracují systémy s~vrstvou \zk{zk:TFT} (\cref{fig:direct-tft}). Matice \zk{zk:TFT} ukládá generovaných náboj do paměťových kondenzátorů, jejichž napětí může být vyčítáno AD převodníkem. Takovýto detektor se skládá z~elektrody na kterou je přivedeno vysoké kladné napětí, polovodičové vrstvy selenu a vrstvy \zk{zk:TFT} kondenzátory pro ukládání náboje generovaného rentgenovým záření. Rentgenové záření generuje v~selenové vrstvě elektrony a díry. Díky tomu vzniká v~polovodiči náboj, který je přímo úměrný rentgenovému rentgenovému záření. Tento náboj je poté díky silnému elektrickému poli odváděn do kondenzátorů v~\zk{zk:TFT} vrstvě. Napětí kondenzátoru může být poté vyčteno přivedení záporného napětí na gate příslušného tranzistoru.

\begin{figure}[ht]
\centering
\includegraphics{direct-tft}
\caption{Princip polovodičového TFT detektoru s~přímým převodem rentgenového záření \cite[str.~511]{Radiation-Detection-and-Measurement}}
\label{fig:direct-tft}
\end{figure}

\paragraph{DR s~nepřímým převodem}
využívá scintilační vrstvy k~převodu rentgenového záření na světelné záření, které je poté snímáno dalšími vrstvami. Scintilační vrstvy můžou být jak strukturované, tak nestrukturované. V~případě nestrukturované scintilační vrstvy generované světelné záření přirozeně dopadá na přilehlé snímače světelného záření, které odpovídají jednotlivým pixelům. Naopak v~případě strukturované scintilační vrstvy světelné záření díky speciální struktuře vrstvy dopadá pouze na jeden snímač světelného záření. Díky tomu je zajištěno lepší prostorové rozlišení detektoru. Rozdíly mezi strukturovanou a nestrukturovanou scintilační vrstvou popisuje \cref{fig:structured-unstructured-scintillator}.

\begin{figure}[ht]
\centering
\includegraphics{structured-unstructured-scintillator}
\caption{Rozdíl mezi nesturkturovanou (A) a strukturovanou (B) scintilační vrstvou TFT detektoru s~nepřímým převodem. \cite[str~210]{Diagnostic-Radiology}}
\label{fig:structured-unstructured-scintillator}
\end{figure}

V~případě, že je jako detekční vrstva využíváno \zk{zk:TFT} matice, musí být pod scintilační vrstvu přidána vrstva detekující světelné záření. Taková vrstva je nejčastěji tvořena foto-diodovým polem. Elektrická energie generována foto-diodovým polem je poté ukládána pomocí kondenzátorů \zk{zk:TFT} matice a následně  čtena AD převodníkem.

Detekční vrstva může být také realizována pomocí \zkratka{zk:CCD}. V~tomto případě je pomocí \zk{zk:CCD} detektorů přímo detekováno viditelné světlo, které je generováno v~scintilační vrstvě. Díky poměrně malé velikosti \zk{zk:CCD} snímačů je nutné snímače kombinovat do matic, případně využívat čoček.

\section{Metody předzpracování rentgenových snímků}

\subsection{Reprezentace digitálního snímku}
Pořízený digitální snímek může být reprezentován pomocí signálu v~prostorové nebo frekvenční oblasti. Způsob reprezentace pořízeného obrázku se odvíjí především od způsobu jeho pořizování. V~případě digitální radiografie jsou snímky pořizovány v~prostorové oblasti. Snímky ve frekvenční oblasti lze například nalézt u~magnetické rezonance -- Magnetic Resonance Inspection (MRI).

\paragraph{Prostorová oblast}
digitálních snímků je reprezentována pomocí matice o~velikosti $M \times N$, kde $M$ je počet řádků a $N$ počet sloupců. Jednotlivé prvky matice se nazývají pixely. Počet sloupců a řádků určuje rozlišení snímku a je udáváno jako $M \times N$ nebo jako výsledek jejich součinu, čili absolutní počet pixelů -- například rozlišení $1024 \times 1024$ lze zapsat jako \num{1048576} pixelů.  Pixely mohou nabývat hodnot od 0 do $2^{n}$, kde číslo $n$ je nazýváno jako bitová hloubka snímků. V~případě digitálních rentgenových snímků bývá datová hloubka nejčastěji 10, 12 nebo 16 bitů, nicméně hodnota bývá zpravidla ukládána ve 2 bytech, tudíž v~případě bitové hloubky 10 nebo 12 bitů zůstávají vyšší bity nulové.

Digitální snímek v~prostorového oblasti se skládá z~kanálů, které jsou reprezentovány pomocí matic o~stejném počtu řádků a sloupců, jako má výsledný digitální snímek. Počet kanálů a to, co reprezentují je určeno barevným modelem. Například jedním z~nejznámějších barevných modelů je model RGB, který definuje 3 barevné kanály: červený (RED), zelený (GREEN) a modrý (BLUE). Každý kanál reprezentuje intenzitu jedné z~barev a jejich sečtením vzniká výsledný snímek. Vzhledem k~způsobu pořizování rentgenových snímků, mají snímky pouze jeden kanál, který zpravidla bývá interpretován pomocí stupňů šedi -- grayscale.

\paragraph{Frekvenční oblast}
digitální snímků reprezentuje frekvence obsažené v~obrázku. Frekvence signálu digitálního snímku odpovídá počtu cyklů na jeden pixel. Například pokud snímek obsahuje řádek, který se skládá z~periodicky se opakující dvojice černého a bílého pixelu, digitální snímek bude obsahovat signál o~frekvenci \SI{1/2}{\hertz}. 

Vzhledem k~tomu, že ve většině případů je pořízený snímek v~prostorové oblasti musí být snímek do frekvenční oblasti převeden. V~případě digitálních snímků bývá pro převod nejčastěji využíváno diskrétní Fourierovy transformace (DFT). Pro zpětný převod je poté využíváno zpětné DFT. DFT je definována jako:
\begin{equation}
\label{eq:dft}
F(u,v)=\sum_{x=0}^{M-1} \sum_{y=0}^{N-1} f(x,y) e^{\frac{-j2\pi ux}{M} - \frac{j2\pi vy}{N}}
\end{equation}
a zpětnou DFT jako:
\begin{equation}
f(x,y)=\frac{1}{MN}\sum_{u=0}^{M-1}\sum_{v=0}^{N-1}F(u,v)e^{\frac{j2\pi ux}{M}+\frac{j2\pi vy}{N}},
\end{equation}
kde $f(n,m)$ představuje konkrétní pixel obrázku v~prostorové oblasti s~indexem $n$ a $m$, $N$ a $M$ jsou rozměry velikost snímku, $F(k,l)$ je reprezentace snímku ve frekvenční oblasti pro různé frekvence reprezentované pomocí $k$ a $l$. Velikost $k$ a $l$ se pohybuje v~rozsahu od 0 do $M-1$ a od 0 do $N-1$
, přičemž $F(0,0)$ udává velikost stejnosměrné složky obrázku a $F(M-1,N-1)$ udává velikost složky s~nejvyšší možnou frekvencí. Normalizační člen zpětné DFT $\frac{1}{MN}$ může být přesunuta do rovnice \ref{eq:dft}, kdy pro $F(0,0)$ je poté stejnosměrná složka rovna průměrné hodnotě pixelů snímku.

\subsection{Metody předzpracování rentgenových snímků}
Operace prováděné při předzpracování digitálních RTG snímků lze dle Baxese \cite{baxes} a Gonzáleze \cite{Gonzalez} rozdělit do pěti základních kategorií. Tyto kategorie zahrnují:
\begin{itemize}
\item Obnovení -- obnovení poškozených snímků například kompresí, rozmazáním či ztrátou dat.
\item Analýzu -- operace zahrnující segmentaci, klasifikaci apod. 
\item Syntézu -- sloučení více snímků pro získání více informací. Sloučením může vzniknout například nový snímek nebo 3D model v~případě \zkratka{zk:CT}.
\item Zvýšení kvality -- operace sloužící pro zvýšení kvality za účelem zvýšení diagnostické hodnoty snímku. Nejčastěji se jedná o~zvýraznění hran, úpravu kontrastu apod.
\item Kompresi -- snížení velikosti snímků.
\end{itemize}

V~případě digitálních rentgenových snímků a metod jejich předzpracování se můžeme setkat s~operacemi ze všech kategorií. V~případě operací pro obnovení se nejčastěji setkáváme s~odstraněním šumu který vzniká díky nehomogenitě generovaného rentgenového záření. Operace pro analýzu zahrnují především segmentaci obrazu a klasifikaci. V~případě operací pro zvyšování kvality je v~radiologii využíváno především zvýrazňování hran, úpravy kontrastu a dynamického rozsahu. Kompresní operace jsou využívány především při ukládání a archivaci digitálních snímků, kdy je kladen důraz na snížení velikosti digitálních snímků při zachování jejich dostatečné kvality.

\subsubsection{Odstranění šumu}
Filtrováním digitálních rentgenových snímků se rozumí provádění operací za účelem zvýšení jejich kvality, především z~hlediska rozlišení, kontrastu a šumu. V~případě filtrování v~prostorové oblasti jsou operace prováděny postupně na jeden nebo skupinu pixelů. \cite[str.~425]{Diagnostic-Radiology-Physics}
\paragraph{Průměrování}
je jedním z~nejjednodušších filtrů. Funguje na principu nahrazení hodnoty pixelu hodnotou průměru jeho okolí o~velikosti $N$. Díky nahrazení hodnoty pixelu průměrem jeho okolí dochází k~filtrování složek s~vysokou frekvencí ve frekvenční oblasti. Vliv filtru a změny velikosti jeho okolí ukazuje \cref{fig:mean-filter}. Z~obrázku je zřejmé, že filtr filtruje vysokofrekvenční složky -- tmavá linka ve spodní části obrázku a čáry na obloze. Matematicky lze průměrovací filtr vyjádřit pomocí konvoluce následovně:

\begin{equation}
\label{eq:mean1}
I_{filtered} = I \ast K, \quad
K = \frac{1}{N^{2}}\begin{bmatrix}
 &1  &1 &\dots  &1 \\ 
 &1  &1 &\dots  &1 \\ 
 &\vdots  &\vdots &\ddots  &\vdots \\ 
 &1  &1 &\dots  &1 
\end{bmatrix},
\end{equation}
kde $N$ určuje velikost okolí -- velikost konvoluční matice $N\times N$, $I$ filtrovaný snímek a $I_{filtered}$ snímek po aplikaci filtru.

\begin{figure}[htb]
\centering
\includegraphics{mean-filter}
\caption{Původní snímek (A) a snímky na které je aplikován průměrovací filtr s~okolím $\protect N=10$ (B) a okolím $\protect N=15$ (C).}
\label{fig:mean-filter}
\end{figure}

Ve vztahu k~radiologii průměrovací filtr dokáže vyhladit snímek díky odstranění šumu s~vysokou frekvencí. Mimo odstranění šumu je průměrovací filtr využíván před dalšími kroky předzpracování digitálních rentgenových snímků. \cite{Diagnostic-Radiology-Physics}

\paragraph{Gaussův filtr}
lze stejně jako průměrovací filtr definovat pomocí konvoluce a stejně jako průměrovací filtr se chová jako dolnopropustní filtr -- filtruje vysokofrekvenční složky. V~případě Gaussova filtru však konvoluční matice nezajišťuje průměrování, ale je vygenerována na základě aproximace Gaussova rozložení dle rovnice \cite[str.~139]{Image-Processing-Analysis-and-Machine-Vision}:

\begin{equation}
\label{eq:gaussian1}
G(x,y)=e^{\frac{-(x^2+y^2)}{2\sigma^2}}
\end{equation}
kde $x$ a $y$ jsou souřadnice a $\sigma$ směrodatná odchylka. Často bývá rovnice \ref{eq:gaussian1} rozšiřována o~normalizační členy:
\begin{align}
\label{eq:gaussian2}
G(x,y)&= \frac{1}{2\pi\sigma^2} e^{\frac{-(x^2+y^2)}{2\sigma^2}} \quad a
\\
G(x,y)&= \frac{1}{\sqrt{2\pi}\sigma} e^{\frac{-(x^2+y^2)}{2\sigma^2}}
\end{align}

Vzhledem k~tomu, že normální rozložení lze definovat pro všechna reálná čísla, výsledná matice by byla nekonečně veliká, proto se v~praxi často velikost matice omezuje tak, aby matice obsahovala všechny hodnoty normálního rozložení v~intervalu $<-3\sigma, 3\sigma>$. Aplikaci Gaussova filtru na reálný snímek popisuje \cref{fig:gaussian-filter}. Stejně jako u~průměrovacího filtru je z~obrázku zřejmá dolnopropustní charakteristika Gaussova filtru.

\begin{figure}[htb]
\centering
\includegraphics{gaussian-filter}
\caption{Původní snímek (A) a snímky na které je aplikován Gaussův filtr s~parametrem $\protect\sigma=3$ (B) a $\protect\sigma=5$ (C).}
\label{fig:gaussian-filter}
\end{figure}

\paragraph{Mediánový filtr} nahrazuje každý pixel snímku mediánem hodnot v~okolí pixelu $N \times N$. Medián hodnot jako takový je nelineární operace, proto ho není možné reprezentovat pomocí konvoluce jako v~případě průměrovacího a Gaussova filtru. Mediánový filtr je využívám především k~odstranění impulsového šumu. Jinými slovy šumu, který způsobuje, že některé izolované pixely mají velmi nízkou nebo naopak vysokou hodnotu oproti svému okolí. Nevýhodou tohoto filtru je, že může odstranit pro nás důležité informace, jako tenké hrany apod. Filtraci snímku, který je uměle zašuměný náhodným šumem typu "salt \and pepper" $20\%$ pokrytím snímku, ukazuje \cref{fig:median-filter}.

\begin{figure}[htb]
\centering
\includegraphics{median-filter}
\caption{Snímek zašumělý pomocí náhodného šumu "salt \and pepper" (A) a snímky na které je aplikován mediánový filtr s~parametrem okolí $\protect N=3$ (B) a $\protect N=5$ (C).}
\label{fig:median-filter}
\end{figure}
\clearpage
\subsubsection{Detekce hran}
Jednou z hlavních aplikací předzpracování rentgenových snímků je detekce oblasti zájmů pro usnadnění analýzy snímků radiologem. Oblasti zájmů jsou v mnoha případech ohraničeny nespojitostmi ve snímku -- hranami. Jako příklad lze uvézt analyzování přítomnosti nežádoucího objektu s vyšší absorpcí rentgenového záření než absorpcí rentgenového záření analyzovaného objektu. V případě, že se nežádoucího objekt nachází v analyzovaném objektu, vznikají na hranicích nežádoucího objektu hrany.

Pro detekci hran je využíváno gradientních algoritmů, které jsou ze své podstaty citlivé na náhlé změny signálu. Gradient pro dvourozměrný signál $\bigtriangledown f(x,y)$ se často vyjadřuje v dvourozměrných souřadnicích jako \cite[str~135]{Image-Processing-Analysis-and-Machine-Vision}:

\begin{align}
\label{eq:grad2D1}
\vert \bigtriangledown f(x,y)\vert&=\sqrt{\left(\frac{\partial f}{\partial x}\right)^2\left(\frac{\partial f}{\partial y}\right)^2},
\\
\psi &= \arg\left(\frac{\partial f}{\partial x}, \frac{\partial f}{\partial y}\right)
\end{align}

kde $x$ a $y$ jsou indexy pixelů ve snímku, $\psi$ je směr gradientu v radiánech a $\vert \bigtriangledown f(x,y)\vert$ velikost gradientu. V případě, že nás zajímá pouze velikost hrany můžeme použít Laplaceův lineární diferenční operátor, který lze vyjádřit jako:

\begin{align}
\label{eq:laplac}
\bigtriangledown ^2 f(x,y) = \frac{\partial^2f(x,y)}{\partial x^2}+\frac{\partial^2f(x,y)}{\partial y^2}
\end{align}

Gradientní operátory, které dokáží hledat hrany lze rozdělit dle Sonky \cite{Image-Processing-Analysis-and-Machine-Vision} do tří kategorií na operátory založené na:
\begin{itemize}
\item Aproximaci derivace dvourozměrného signálu snímku.
\item Průchodu druhých derivací nulou (zero-crossing).
\item Porovnávání dvourozměrného signálu snímku s parametrickým modelem hran.
\end{itemize}
Gradientní operátory, které jsou aplikovány na malé části digitálního snímku, jsou ve své podstatě konvoluce, které lze definovat pomocí konvoluční masky.

\paragraph{Robertsův operátor} 
je jedním z nejjednodušších operátorů, který je je definován pomocí dvou dvourozměrných konvolučních masek: 
\begin{equation}
h1=
\begin{bmatrix}
1 & -0\\ 
0 & -1
\end{bmatrix},  \quad
h2=
\begin{bmatrix}
 0 & 1\\ 
-1 & 0
\end{bmatrix} 
\end{equation}
Díky malému okolí je detekce hran pomocí toho operátoru málo náročné na výpočet, na druhou stranu díky svému malému okolí pro aproximaci gradientu je velmi citlivý na šum. \cite[]str.~135{Image-Processing-Analysis-and-Machine-Vision}

\paragraph{Laplaceův operátor}
$\bigtriangledown ^2$ je oblíbený operátor aproximující druhou derivaci, jejíž výsledkem je pouze pouze velikost hrany. Rovnice \ref{eq:laplac} je v případě digitálních snímků aproximován pomocí součtu konvolucí. Nevýhodou Laplaceova operátoru je, že v některých případech může reagovat dvakrát na stejnou hranu. Jako aproximace pro okolí 4 nebo 8 je často využívána maska $3\times 3$ \cite[str.~136]{Image-Processing-Analysis-and-Machine-Vision}
\begin{equation}
h=
\begin{bmatrix}
 0 & -1 &  0\\
 1 & -4 &  1\\ 
 0 &  1 &  0\\ 
\end{bmatrix},  \quad
h=
\begin{bmatrix}
 1 &  1 &  1\\
 1 & -8 &  1\\ 
 1 &  1 &  1\\ 
\end{bmatrix}
\end{equation}
V případě, že je vyžadováno zvýraznění středního pixelu je využita aproximace pomocí masky:
\begin{equation}
h=
\begin{bmatrix}
-1 &  2 & -1\\
 2 & -4 &  2\\ 
-1 &  2 & -1\\ 
\end{bmatrix},  \quad
h=
\begin{bmatrix}
 2 & -1 &  2\\
-1 & -4 & -1\\ 
 2 & -1 &  2\\ 
\end{bmatrix}
\end{equation}

\paragraph{Prewittův operátor} aproximuje první derivaci. Gradient je aproximován konvolucí ve všech možných směrech. Směr pro který je výsledek konvoluce největší poté určí směr gradientu. Jako aproximující konvoluční matici můžeme zvolit například matici $3\times 3$, kterou poté rotujeme:
\begin{equation}
h_{1}=
\begin{bmatrix}
 1 &  1 &  1\\
 0 &  0 &  0\\ 
-1 & -1 & -1\\ 
\end{bmatrix},  \quad
h_{2}=
\begin{bmatrix}
 0 &  1 &  1\\
-1 &  0 &  1\\ 
-1 & -1 &  0\\ 
\end{bmatrix},  \quad
h_{3}=
\begin{bmatrix}
-1 &  0 &  1\\
-1 &  0 &  1\\ 
-1 &  0 &  1\\ 
\end{bmatrix},\quad ...
\end{equation}

\paragraph{Sobelův operátor} stejně jako Prewittův a Robertsonvů operátor aproximuje první derivaci. Tento operátor je často využíván jako detektor vodorovnosti a horizontálnosti hran. Míra horizontálnosti a vertikálnosti je dána výsledkem konvoluce s konvolučními maskami, které jsou dány rovnicí \ref{eq:sobel}. Uvážíme-li výsledek konvoluce s jádrem $h_{1}$ jako $y$ a výsledek konvoluce s jádrem $h_{2}$ jako x, tak lze velikost hrany vypočítat jako $\sqrt{x^2+y^2}$ nebo $\left| x\right| + \left| y\right|$. Směr hrany potom se poté určí jako $\arctan \left( \frac{y}{x} \right)$. \cite[str.~137]{Image-Processing-Analysis-and-Machine-Vision}

\begin{equation}
\label{eq:sobel}
h_{1}=
\begin{bmatrix}
 1 &  2 &  1\\
 0 &  0 &  0\\ 
-1 & -2 & -1\\ 
\end{bmatrix},  \quad
h_{2}=
\begin{bmatrix}
 0 &  1 &  2\\
-1 &  0 &  1\\ 
-2 & -1 &  0\\ 
\end{bmatrix}
\end{equation}

\begin{figure}[ht]
\centering
\includegraphics{edge-detectors1}
\caption{Aplikace gradientních metod pro detekci hran na rentgenový snímek.}
\label{fig:edge-detector1}
\end{figure}

\begin{figure}[ht]
\centering
\includegraphics{edge-detectors2}
\caption{Aplikace gradientních metod pro detekci hran na rentgenový snímek.}
\label{fig:edge-detector2}
\end{figure}

\begin{figure}[ht]
\centering
\includegraphics{edge-detectors3}
\caption{Aplikace gradientních metod pro detekci hran na rentgenový snímek.}
\label{fig:edge-detector3}
\end{figure}

\clearpage

\paragraph{Detektor průchodu nulou} je založen na principu hledání průchodu nulou v druhé derivaci signálu -- snímku. Uvážíme-li hranu v jednorozměrném signálu, výsledkem první derivace tohoto signálu bude extrém v místě hrany. Po provedení druhé derivace bude výsledek v místě hrany procházet nulou. Této vlastnosti je využíváno především proto, že je jednodušší najít průchod funkce nulou než její extrémy. Vliv derivace na signál s hranou popisuje \cref{fig:zero-crossing-principle}.

\begin{figure}[ht]
\centering
\includegraphics[width=\textwidth]{zero-crossing-principle}
\caption{Vliv první a druhé derivace na jednorozměrný signál -- princip detekce hran průchodu nulou. \cite[str.~138]{Image-Processing-Analysis-and-Machine-Vision}}
\label{fig:zero-crossing-principle}
\end{figure}

Hlavním problémem této metody je nalezení robustní metody pro výpočet druhé derivace. Jedním z možných postupů je vyhlazení digitálního snímku Gaussovým filtrem a následné vypočtení druhé derivace vyhlazeného snímku. Gaussův vyhlazovací filtr byl popsán v předchozích částech práce a je definován rovnicemi \ref{eq:gaussian1} a \ref{eq:gaussian2}. K výpočtu druhé derivaci lze využít Laplaceova operátoru, který je definovaný rovnicí \ref{eq:laplac}.  Operace při které je aplikován Laplaceův operátor $\bigtriangledown ^2$ na snímek $f(x,y)$, který je vyhlazen Gaussovým filtrem $G(x, y,\sigma )$ je nazývána anglickým spojením Laplacian of Gaussian (LoG). LoG je definována  jako \cite[str.~139]{Image-Processing-Analysis-and-Machine-Vision}:

\begin{align}
\label{eq:LoG1}
LoG \left (f \left (x,y \right ) \right )=
\bigtriangledown^2\left [ G(x,y,\sigma)\ast f(x,y) \right ] = 
\left [\bigtriangledown^2G(x,y,\sigma) \right ] \ast f(x,y)
\end{align}

Derivace Gaussova filtru $\bigtriangledown^2G(x,y,\sigma)$ může být analyticky vypočítán před aplikací, díky čemuž je snížena obtížnost výpočtu. Zavedeme-li substituci $r^2=x^2+y^2$, kde $r$ určuje vzdálenost od počátku. Díky této substituci můžeme první a druhou derivaci Gaussova filtru počítat jako derivaci funkce o jedné proměnné \cite[str.~140]{Image-Processing-Analysis-and-Machine-Vision}:
\begin{align}
\label{eq:LoG2}
G(r)    &=e^{\frac{-r^2}{2\sigma^2}} \\
G{}'(r) &= -\frac{1}{\sigma^2} r e^{\frac{-r^2}{2\sigma^2}} \\
G{}''(r)&= \frac{1}{\sigma^2} \left ( \frac{r^2}{\sigma^2} - 1 \right ) e^{\frac{-r^2}{2\sigma^2}} 
\end{align}
Po nahrazení substituce původním výrazem $x^2 + y^2$ a přidáním normalizačního koeficientu $c$ je výsledná rovnice pro získání konvoluční masky LoG operátoru:
\begin{align}
\label{eq:LoG2_1}
h\left ( x,y \right )=c \left ( \frac{x^2+y^2-\sigma^2}{\sigma^4} \right ) e^{\frac{-(x^2+y^2)}{2\sigma^2}}, 
\end{align}
kde $c$ je normalizační keoficient, který zajišťuje, aby součet všech prvků výsledné konvoluční matice byl 0. \cite[str.~140]{Image-Processing-Analysis-and-Machine-Vision}

\subsubsection{Segmentace}
