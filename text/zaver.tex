\chapter{Závěr}

Cílem této práce bylo na základě literární rešerše metod pro předzpracování rentgenových snímků a literární rešerše základních principů pořizování rentgenových snímků navrhnout a realizovat metody pro předzpracování série snímků za účelem odstranění šumu, zvýšení dynamického rozsahu či zvětšení rozlišení. Dalším cílem práce bylo vytvořit navrhnout a realizovat software ve formě knihovny pro ukládání digitálních rentgenových snímků. Tato realizace byla provedena na základě literární rešerše dostupných datových formátů pro ukládání digitálních rentgenových snímků.

V~první části práce byla provedena literární rešerše principů pořizování rentgenových snímků. Tato rešerše popisuje vznik rentgenového záření a způsob, jakým je rentgenové záření generováno v~rentgenkách. V~další části rešerše je popsán způsob detekce rentgenového záření. Tato část se zabývá především konstrukcí detektorů rentgenového záření a jejího vlivu na výsledný snímek.

Následující kapitola se zabývala obecnými principy předzpracování digitálních rentgenových snímků. V~této kapitole byly nejprve popsány základní principy reprezentace obrazu a následně definovány oblasti předzpracování digitálních rentgenových snímků. Závěrečná část této kapitoly se zabývala popisem metod předzpracování rentgenových snímků. V~rámci kapitoly byly všechny metody implementovány pomocí nástroje Jupyter notebook v~jazyce Python. Výstupy implementace byly poté použity pro ilustraci fungování metod pro předzpracování rentgenových snímků.

Navazující kapitola se zabývala předzpracováním série rentgenových snímků. V kapitole byly popsány a implementovány dvě metody pro odstranění šumu průměrováním. První z metod prováděla jednoduché průměrování a druhá z metod ještě před samotným průměrováním prováděla obrazovou registraci metodou ORB na základě které byl vytvořen transformační model metodou RANSAC. Pomocí transformačního modelu byly všechny snímky přeneseny do stejné souřadnicové soustavy. Implementace obou metod byla ověřena na uměle zašumělých rentgenových snímcích. Na konci kapitoly byly poté popsány výsledky obou metod, kdy se prokázalo, že metoda s obrazovou registrací je odolnější vůči šumu způsobeného posunem objektu ve snímku.

Další část této kapitoly se zabývala zvýšením dynamického rozsahu sloučením série rentgenových snímků. Jako metoda pro zvýšení dynamického rozsahu byla využita Debevecova metoda pro generování HDR snímku, která se používá při slučování běžných fotografických snímků. V rámci kapitoly byla tato metoda stručně popsána a následně implementována. Po aplikaci implementace na sérii testovacích snímků bylo ověřeno, že Debevecovu metodu lze aplikovat i na sérii rentgenových snímků. Jak metody pro odstranění šumu, tak metody pro zvýšení dynamického rozsahu byly stejně jako v předchozí kapitole implementovány pomocí jazyka Python v prostředí Jupyter notebook.

V kapitole práce byla provedena stručná rešerše datových formátů pro ukládání rentgneových snímků. Tato kapitola se zaměřila na popis standardu DICOM a jeho derivátu DICONDE. Kapitola popsala základní strukturu formátu a jeho vlastnosti. Na konci této kapitoly byl vybrán vhodný software ve formě knihovny, díky kterému lze pracovat s tímto datovým formátem, potažmo standardem.