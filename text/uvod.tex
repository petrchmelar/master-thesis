\chapter*{Úvod} 
\phantomsection 
\addcontentsline{toc}{chapter}{Úvod} 
Tato práce se zabývá předzpracováním digitálních rentgenových snímků a jejich ukládáním. Oblast předzpracování rentgenových snímků je velmi důležitá především v oborech, kde je potřebná nedestruktivní rentgenová defektoskopie. Mezi tyto obory patří například lékařská a průmyslová radiologie. Předzpracování rentgenových snímků je v těchto oblastech důležité zejména díky možnosti odstranění nežádoucích vlivů -- především šumu a zvýraznění důležitých informací ve snímku. Jedním z problémů metod pro předzpracování jednotlivých rentgenových snímků je, že pomocí žádné z dostupných metod pro předzpracování nelze získat více informací, než daný snímek obsahuje.

Cílem práce je vytvoření literární rešerše způsobu získávání a metod předzpracování rentgenových snímků na základě které je možné navrhnout a realizovat metody pro předzpracování série rentgenových snímků. Zpracováním série rentgenových snímků v jeden snímek je možná získat více informací než v případě jednoho snímku. Tato práce se zaměřuje především na metody předzpracování série snímků za účelem odstranění šumu, zvýšení dynamického rozsahu či zvýšení rozlišení výsledného snímku. Ověření funkčnosti realizovaných algoritmů je provedena díky nasnímané množině rentgenových snímků.

Mimo realizaci metod pro předzpracování rentgenových snímků se tato práce zabývá rešerší datových formátů, případně standardů pro ukládání rentgenových snímků. Na základě této rešerše je zvolen vhodný formát, který je následně implementován. Implementace softwaru pro práci s datovým formátem je provedena ve formě knihovny pro snadnou implementaci v komplexnějších aplikacích.