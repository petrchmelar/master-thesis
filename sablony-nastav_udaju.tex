%% Název práce:
%  První parametr je název v originálním jazyce,
%  druhý je překlad v angličtině nebo češtině (pokud je originální jazyk angličtina)
\nazev{Nástroje pro předzpracování rentgenových snímků}{Radiography preprocessing tools}

%% Jméno a příjmení autora ve tvaru
%  [tituly před jménem]{Křestní}{Příjmení}[tituly za jménem]
\autor[Bc.]{Petr}{Chmelař}

%% Jméno a příjmení vedoucího/školitele včetně titulů
%  [tituly před jménem]{Křestní}{Příjmení}[tituly za jménem]
% Pokud osoba nemá titul za jménem, smažte celý řetězec '[...]'
\vedouci[Ing.]{Petr}{Petyovský}[Ph.D.]

%% Jméno a příjmení oponenta včetně titulů
%  [tituly před jménem]{Křestní}{Příjmení}[tituly za jménem]
% Pokud nemá titul za jménem, smažte celý řetězec '[...]'
% Uplatní se pouze v prezentaci k obhajobě;
% v případě, že nechcete, aby se na titulním snímku prezentace zobrazoval oponent, pouze jej zakomentujte;
% u obhajoby semestrální práce se oponent nezobrazuje
\oponent[doc.\ Mgr.]{Křestní}{Příjmení}[Ph.D.]

%% Označení oboru studia
% První parametr je obor v originálním jazyce,
% druhý parametr je překlad v angličtině nebo češtině
\oborstudia{Kybernetika, automatizace a měření}{Cybernetics, Control and Measurements}

%% Označení fakulty
% První parametr je název fakulty v originálním jazyce,
% druhý parametr je překlad v angličtině nebo v češtině
%\fakulta{Fakulta architektury}{Faculty of Architecture}
\fakulta{Fakulta elektrotechniky a komunikačních technologií}{Faculty of Electrical Engineering and Communication}
%\fakulta{Fakulta chemická}{Faculty of Chemistry}
%\fakulta{Fakulta informačních technologií}{Faculty of Information Technology}
%\fakulta{Fakulta podnikatelská}{Faculty of Business and Management}
%\fakulta{Fakulta stavební}{Faculty of Civil Engineering}
%\fakulta{Fakulta strojního inženýrství}{Faculty of Mechanical Engineering}
%\fakulta{Fakulta výtvarných umění}{Faculty of Fine Arts}

%% Označení ústavu
% První parametr je název ústavu v originálním jazyce,
% druhý parametr je překlad v angličtině nebo češtině
\ustav{Ústav automatizace a měřicí techniky}{Department of Control and Instrumentation}
%\ustav{Ústav biomedicínského inženýrství}{Department of Biomedical Engineering}
%\ustav{Ústav elektroenergetiky}{Department of Electrical Power Engineering}
%\ustav{Ústav elektrotechnologie}{Department of Electrical and Electronic Technology}
%\ustav{Ústav fyziky}{Department of Physics}
%\ustav{Ústav jazyků}{Department of Foreign Languages}
%\ustav{Ústav matematiky}{Department of Mathematics}
%\ustav{Ústav mikroelektroniky}{Department of Microelectronics}
%\ustav{Ústav radioelektroniky}{Department of Radio Electronics}
%\ustav{Ústav teoretické a experimentální elektrotechniky}{Department of Theoretical and Experimental Electrical Engineering}
%\ustav{Ústav telekomunikací}{Department of Telecommunications}
%\ustav{Ústav výkonové elektrotechniky a elektroniky}{Department of Power Electrical and Electronic Engineering}

\logofakulta[loga/FEKT_zkratka_barevne_PANTONE_CZ]{loga/UTKO_color_PANTONE_CZ}


%% Rok obhajoby
\rok{Rok}
\datum{1.\,1.\,1970} % Datum se uplatní pouze v prezentaci k obhajobě

%% Místo obhajoby
% Na titulních stránkách bude automaticky vysázeno VELKÝMI písmeny
\misto{Brno}

%% Abstrakt
\abstrakt{
Tato práce se zabývá návrhem a realizací metod předzpracování rentgenových snímků.
V~řešení byly na základě literární rešerše současných metod navrženy a realizovány algoritmy pro předzpracování série rentgenových snímků, jako průměrování po obrazové registraci nebo sloučení snímků v HDR snímek pomocí Debevecovy metody.
V~další části řešení byla provedena literární rešerše datových formátů pro ukládání rentgenových snímků a proveden výběr vhodného datového formátu.
}{
This thesis deals with design and realization of methods of preprocessing of X-ray images.
In the solution part of this thesis, there were designed and implemented methods for preprocessing of series of X-ray images such as averaging after image registration or creation of HDR image using Debevec method.
In the following part of the solution of the thesis, there was done a literary research of data formats of X-ray image storage and suitable data format was selected.
}

%% Klíčová slova
\klicovaslova{
předzpracování rentgenových snímků,
radiografie, 
HDR,
snižování šumu,
zvyšování rozlišení,
DICONDE,
DICOM
}{
X-ray image preprocessing,
radiography,
HDR,
noise reduction,
resolution enhancement,
DICONDE,
DICOM
}

%% Poděkování
\podekovanitext{Rád bych poděkoval vedoucímu semestrální práce panu Ing.~Petru Petyovskému, Ph.D.\ za odborné vedení, konzultace, trpělivost a podnětné návrhy k~práci.}